\section{Dalitz\-Smear Class Reference}
\label{classDalitzSmear}\index{DalitzSmear@{DalitzSmear}}
Class that takes in a 4vector and smears it, each vector both smeared and unsmeared is stored locally, and the class requires the particle id to be seperately supplied to the class (where it is also stored locally).  


{\tt \#include $<$Dalitz\-Smear.h$>$}

\subsection*{Public Member Functions}
\begin{CompactItemize}
\item 
void \bf{setpid} (int id)
\begin{CompactList}\small\item\em Method to set the particle ID. \item\end{CompactList}\item 
\bf{Dalitz\-Smear} ()
\begin{CompactList}\small\item\em Empty constructor. \item\end{CompactList}\item 
\bf{Dalitz\-Smear} (TLorentz\-Vector v, int classifier)
\begin{CompactList}\small\item\em Constructor that sets the 4 vector to be smeared, and the particle ID associated with the 4 vector, highly recommended to use this constructor over the empty version. \item\end{CompactList}\item 
TLorentz\-Vector \bf{Smear\-Vector} ()
\begin{CompactList}\small\item\em Method that smears a 4 vector $\ast$NOTE$\ast$ v\_\-reg and pid global variables must already be set. \item\end{CompactList}\item 
TLorentz\-Vector \bf{Smear\-Vector} (TLorentz\-Vector v)
\begin{CompactList}\small\item\em Method that smears the argument 4 vector $\ast$NOTE$\ast$ the particle ID must be set before calling this method. \item\end{CompactList}\item 
TLorentz\-Vector \bf{smear\-Direction} (TLorentz\-Vector v)
\begin{CompactList}\small\item\em Method that initiates the sequence of private calls that create a smeared vector v which is the direction smearing of unit vector u contained in input Tlorentz\-Vector v. \item\end{CompactList}\item 
void \bf{set\-Scale\-Parameter\-NP} (double sigma)
\begin{CompactList}\small\item\em method to set the scale parameter for neutral particle i.e. photon, default value is 1e-4 \item\end{CompactList}\item 
void \bf{set\-Scale\-Parameter\-CP} (double sigma)
\begin{CompactList}\small\item\em method to set the scale parameter for charged particle i.e. positron/electron, default value is 1e-6 \item\end{CompactList}\end{CompactItemize}
\subsection*{Public Attributes}
\begin{CompactItemize}
\item 
TRandom1 $\ast$ \bf{RNG}
\begin{CompactList}\small\item\em Local copy of random number generator used for generating gaussian values. \item\end{CompactList}\item 
TLorentz\-Vector \bf{v\_\-reg}
\begin{CompactList}\small\item\em Locally stored 4 vector which is unsmeared. \item\end{CompactList}\item 
TLorentz\-Vector \bf{v\_\-sme}
\begin{CompactList}\small\item\em Locally stored 4 vector which is the smeared version of v\_\-reg. \item\end{CompactList}\item 
double \bf{q}
\begin{CompactList}\small\item\em Charge, used only electron(-1) and positron(+1). \item\end{CompactList}\item 
int \bf{pid}
\begin{CompactList}\small\item\em The local particle ID associated with the current copy of v\_\-reg. \item\end{CompactList}\end{CompactItemize}
\subsection*{Private Member Functions}
\begin{CompactItemize}
\item 
double \bf{get\-Ptsm} (TLorentz\-Vector v)
\begin{CompactList}\small\item\em Function that returns the smeared values for Pt, which is used for smearing each component of the v\_\-reg vector. \item\end{CompactList}\item 
double \bf{get\-Gauss} (TLorentz\-Vector v)
\begin{CompactList}\small\item\em Generates a random gaussian value for the input vector. If the associated particle ID is an electron or positron the gaussian mean is the curvature of the particle with its associated sigma. If the associated partice ID is a photon the gaussian mean is the measured photon energy with its associated sigma. \item\end{CompactList}\item 
TVector3 \bf{get\-Unit\-Vector} (double i, double j, double k)
\begin{CompactList}\small\item\em Gives the unit vector from the associated input 3 vector. \item\end{CompactList}\item 
double \bf{get\-Vector\-Magnitude} (double i, double j, double k)
\begin{CompactList}\small\item\em Gives the magnitude vector of the input 3 vector. \item\end{CompactList}\item 
TVector3 \bf{get\-Omega\_\-T} (TVector3 v\_\-1, TVector3 v\_\-n)
\begin{CompactList}\small\item\em calculates the omega\_\-t unit vector, which is orthogonal to the original input vector to be smeared. The omega\_\-t vector dictates the direction on the plane normal to the orignal vector will be smeared. \item\end{CompactList}\item 
TVector3 \bf{get\-V\_\-1} (TVector3 u\-Vector)
\begin{CompactList}\small\item\em finds a vector v1 orthogonal to the original input vector u, and is intended to be a component of omega\_\-t \item\end{CompactList}\item 
double \bf{get\-Scalar\-Product} (TVector3 v1, TVector3 v2)
\begin{CompactList}\small\item\em calculates the dot product between to vectors v1 . v2 \item\end{CompactList}\item 
TVector3 \bf{get\-Vector\-Product} (TVector3 v1, TVector3 v2)
\begin{CompactList}\small\item\em calculates the cross product between to vectors v1 x v2 \item\end{CompactList}\item 
double \bf{get\-Dpsi} (TLorentz\-Vector v)
\begin{CompactList}\small\item\em calculates the angular deviaton dpsi for vector v based on a random deviate drawn from a Rayleigh distribution that depends on scale parameter sigma. The scale parameters selected depend on the particle's PID owned by the class at the time of smearing. i.e. the scale parameters for charged particles and neutral particles should be different. \item\end{CompactList}\end{CompactItemize}
\subsection*{Private Attributes}
\begin{CompactItemize}
\item 
double \bf{scale\-Parameter\-CP}
\begin{CompactList}\small\item\em The scale parameter sigma for a charged particle, for drawing random deviates from rayleigh distribution. \item\end{CompactList}\item 
double \bf{scale\-Parameter\-NP}
\begin{CompactList}\small\item\em The scale parameter sigma for a neutral particle, for drawing random deviates from rayleigh distribution. \item\end{CompactList}\end{CompactItemize}


\subsection{Detailed Description}
Class that takes in a 4vector and smears it, each vector both smeared and unsmeared is stored locally, and the class requires the particle id to be seperately supplied to the class (where it is also stored locally). 



\subsection{Constructor \& Destructor Documentation}
\index{DalitzSmear@{Dalitz\-Smear}!DalitzSmear@{DalitzSmear}}
\index{DalitzSmear@{DalitzSmear}!DalitzSmear@{Dalitz\-Smear}}
\subsubsection{\setlength{\rightskip}{0pt plus 5cm}Dalitz\-Smear::Dalitz\-Smear ()}\label{classDalitzSmear_ee03f5d7721abecd419ec687c392a080}


Empty constructor. 

\index{DalitzSmear@{Dalitz\-Smear}!DalitzSmear@{DalitzSmear}}
\index{DalitzSmear@{DalitzSmear}!DalitzSmear@{Dalitz\-Smear}}
\subsubsection{\setlength{\rightskip}{0pt plus 5cm}Dalitz\-Smear::Dalitz\-Smear (TLorentz\-Vector {\em v}, int {\em classifier})}\label{classDalitzSmear_d76d70d15e07a9264c54a6052458e4d9}


Constructor that sets the 4 vector to be smeared, and the particle ID associated with the 4 vector, highly recommended to use this constructor over the empty version. 



\subsection{Member Function Documentation}
\index{DalitzSmear@{Dalitz\-Smear}!getDpsi@{getDpsi}}
\index{getDpsi@{getDpsi}!DalitzSmear@{Dalitz\-Smear}}
\subsubsection{\setlength{\rightskip}{0pt plus 5cm}double Dalitz\-Smear::get\-Dpsi (TLorentz\-Vector {\em v})\hspace{0.3cm}{\tt  [private]}}\label{classDalitzSmear_63b59045fde437179ab561ea81d369aa}


calculates the angular deviaton dpsi for vector v based on a random deviate drawn from a Rayleigh distribution that depends on scale parameter sigma. The scale parameters selected depend on the particle's PID owned by the class at the time of smearing. i.e. the scale parameters for charged particles and neutral particles should be different. 

\begin{Desc}
\item[Parameters:]
\begin{description}
\item[{\em v}]The four momenta which direction is to be smeared. \end{description}
\end{Desc}
\begin{Desc}
\item[Returns:]angular deviation dpsi \end{Desc}
\index{DalitzSmear@{Dalitz\-Smear}!getGauss@{getGauss}}
\index{getGauss@{getGauss}!DalitzSmear@{Dalitz\-Smear}}
\subsubsection{\setlength{\rightskip}{0pt plus 5cm}double Dalitz\-Smear::get\-Gauss (TLorentz\-Vector {\em v})\hspace{0.3cm}{\tt  [private]}}\label{classDalitzSmear_18fa31b26891f1138adff5dea3a8f558}


Generates a random gaussian value for the input vector. If the associated particle ID is an electron or positron the gaussian mean is the curvature of the particle with its associated sigma. If the associated partice ID is a photon the gaussian mean is the measured photon energy with its associated sigma. 

\begin{Desc}
\item[Parameters:]
\begin{description}
\item[{\em v}]The input \char`\"{}v\_\-reg\char`\"{} 4 vector \end{description}
\end{Desc}
\index{DalitzSmear@{Dalitz\-Smear}!getOmega_T@{getOmega\_\-T}}
\index{getOmega_T@{getOmega\_\-T}!DalitzSmear@{Dalitz\-Smear}}
\subsubsection{\setlength{\rightskip}{0pt plus 5cm}TVector3 Dalitz\-Smear::get\-Omega\_\-T (TVector3 {\em v\_\-1}, TVector3 {\em v\_\-n})\hspace{0.3cm}{\tt  [private]}}\label{classDalitzSmear_ba454a3cdff75924c7b2613813f1f111}


calculates the omega\_\-t unit vector, which is orthogonal to the original input vector to be smeared. The omega\_\-t vector dictates the direction on the plane normal to the orignal vector will be smeared. 

\begin{Desc}
\item[Parameters:]
\begin{description}
\item[{\em v\_\-1}]the quantity returned from get\-V\_\-1 \item[{\em v\_\-n}]a vector that is orthogonal to v\_\-1 and the original unit vector u \end{description}
\end{Desc}
\begin{Desc}
\item[Returns:]the smearing direction unit 3 vector omega\_\-t \end{Desc}
\index{DalitzSmear@{Dalitz\-Smear}!getPtsm@{getPtsm}}
\index{getPtsm@{getPtsm}!DalitzSmear@{Dalitz\-Smear}}
\subsubsection{\setlength{\rightskip}{0pt plus 5cm}double Dalitz\-Smear::get\-Ptsm (TLorentz\-Vector {\em v})\hspace{0.3cm}{\tt  [private]}}\label{classDalitzSmear_811ae82c27b64f7049a3844148348cc9}


Function that returns the smeared values for Pt, which is used for smearing each component of the v\_\-reg vector. 

\begin{Desc}
\item[Parameters:]
\begin{description}
\item[{\em v}]The input \char`\"{}v\_\-reg\char`\"{} 4 vector \end{description}
\end{Desc}
\index{DalitzSmear@{Dalitz\-Smear}!getScalarProduct@{getScalarProduct}}
\index{getScalarProduct@{getScalarProduct}!DalitzSmear@{Dalitz\-Smear}}
\subsubsection{\setlength{\rightskip}{0pt plus 5cm}double Dalitz\-Smear::get\-Scalar\-Product (TVector3 {\em v1}, TVector3 {\em v2})\hspace{0.3cm}{\tt  [private]}}\label{classDalitzSmear_dfdb5ad9e4c1397965474088bb01c480}


calculates the dot product between to vectors v1 . v2 

\index{DalitzSmear@{Dalitz\-Smear}!getUnitVector@{getUnitVector}}
\index{getUnitVector@{getUnitVector}!DalitzSmear@{Dalitz\-Smear}}
\subsubsection{\setlength{\rightskip}{0pt plus 5cm}TVector3 Dalitz\-Smear::get\-Unit\-Vector (double {\em i}, double {\em j}, double {\em k})\hspace{0.3cm}{\tt  [private]}}\label{classDalitzSmear_c76e17e12e0cc4710c1682bd8b714f86}


Gives the unit vector from the associated input 3 vector. 

\begin{Desc}
\item[Parameters:]
\begin{description}
\item[{\em i}]i component of (i,j,k) \item[{\em j}]j component of (i,j,k) \item[{\em k}]k component of (i,j,k) \end{description}
\end{Desc}
\begin{Desc}
\item[Returns:]the unit vector (i,j,k) \end{Desc}
\index{DalitzSmear@{Dalitz\-Smear}!getV_1@{getV\_\-1}}
\index{getV_1@{getV\_\-1}!DalitzSmear@{Dalitz\-Smear}}
\subsubsection{\setlength{\rightskip}{0pt plus 5cm}TVector3 Dalitz\-Smear::get\-V\_\-1 (TVector3 {\em u\-Vector})\hspace{0.3cm}{\tt  [private]}}\label{classDalitzSmear_63a8df4cff5fd53365540b5d0330b4d4}


finds a vector v1 orthogonal to the original input vector u, and is intended to be a component of omega\_\-t 

\begin{Desc}
\item[Parameters:]
\begin{description}
\item[{\em u\-Vector}]the original, to be smeared, unit vector \end{description}
\end{Desc}
\begin{Desc}
\item[Returns:]the omega\_\-t component v\_\-1 3 vector \end{Desc}
\index{DalitzSmear@{Dalitz\-Smear}!getVectorMagnitude@{getVectorMagnitude}}
\index{getVectorMagnitude@{getVectorMagnitude}!DalitzSmear@{Dalitz\-Smear}}
\subsubsection{\setlength{\rightskip}{0pt plus 5cm}double Dalitz\-Smear::get\-Vector\-Magnitude (double {\em i}, double {\em j}, double {\em k})\hspace{0.3cm}{\tt  [private]}}\label{classDalitzSmear_3cb72fe8ffc410b85b1d9c451954448a}


Gives the magnitude vector of the input 3 vector. 

\begin{Desc}
\item[Parameters:]
\begin{description}
\item[{\em i}]i component of (i,j,k) \item[{\em j}]j component of (i,j,k) \item[{\em k}]k component of (i,j,k) \end{description}
\end{Desc}
\begin{Desc}
\item[Returns:]vector magnitude \end{Desc}
\index{DalitzSmear@{Dalitz\-Smear}!getVectorProduct@{getVectorProduct}}
\index{getVectorProduct@{getVectorProduct}!DalitzSmear@{Dalitz\-Smear}}
\subsubsection{\setlength{\rightskip}{0pt plus 5cm}TVector3 Dalitz\-Smear::get\-Vector\-Product (TVector3 {\em v1}, TVector3 {\em v2})\hspace{0.3cm}{\tt  [private]}}\label{classDalitzSmear_f21d38f7a616dcb5b865e873d64201f7}


calculates the cross product between to vectors v1 x v2 

\index{DalitzSmear@{Dalitz\-Smear}!setpid@{setpid}}
\index{setpid@{setpid}!DalitzSmear@{Dalitz\-Smear}}
\subsubsection{\setlength{\rightskip}{0pt plus 5cm}void Dalitz\-Smear::setpid (int {\em id})}\label{classDalitzSmear_63e8f796d9c84111327341c3eefec43a}


Method to set the particle ID. 

\begin{Desc}
\item[Parameters:]
\begin{description}
\item[{\em id}]the particle ID associated with the 4 vector that is intended to be smeared \end{description}
\end{Desc}
\index{DalitzSmear@{Dalitz\-Smear}!setScaleParameterCP@{setScaleParameterCP}}
\index{setScaleParameterCP@{setScaleParameterCP}!DalitzSmear@{Dalitz\-Smear}}
\subsubsection{\setlength{\rightskip}{0pt plus 5cm}void Dalitz\-Smear::set\-Scale\-Parameter\-CP (double {\em sigma})}\label{classDalitzSmear_811167aba8628dd4812c3f8658bee112}


method to set the scale parameter for charged particle i.e. positron/electron, default value is 1e-6 

\index{DalitzSmear@{Dalitz\-Smear}!setScaleParameterNP@{setScaleParameterNP}}
\index{setScaleParameterNP@{setScaleParameterNP}!DalitzSmear@{Dalitz\-Smear}}
\subsubsection{\setlength{\rightskip}{0pt plus 5cm}void Dalitz\-Smear::set\-Scale\-Parameter\-NP (double {\em sigma})}\label{classDalitzSmear_1713a1dab553950ee81262bec7affa0d}


method to set the scale parameter for neutral particle i.e. photon, default value is 1e-4 

\index{DalitzSmear@{Dalitz\-Smear}!smearDirection@{smearDirection}}
\index{smearDirection@{smearDirection}!DalitzSmear@{Dalitz\-Smear}}
\subsubsection{\setlength{\rightskip}{0pt plus 5cm}TLorentz\-Vector Dalitz\-Smear::smear\-Direction (TLorentz\-Vector {\em v})}\label{classDalitzSmear_581c50a4d6c7694f1d59294e09a9201d}


Method that initiates the sequence of private calls that create a smeared vector v which is the direction smearing of unit vector u contained in input Tlorentz\-Vector v. 

\begin{Desc}
\item[Parameters:]
\begin{description}
\item[{\em v}]the input four momenta to have direction smeared \end{description}
\end{Desc}
\begin{Desc}
\item[Returns:]direction smeared 4 vector \end{Desc}
\index{DalitzSmear@{Dalitz\-Smear}!SmearVector@{SmearVector}}
\index{SmearVector@{SmearVector}!DalitzSmear@{Dalitz\-Smear}}
\subsubsection{\setlength{\rightskip}{0pt plus 5cm}TLorentz\-Vector Dalitz\-Smear::Smear\-Vector (TLorentz\-Vector {\em v})}\label{classDalitzSmear_ae475bf2bea39d1ff580ad0e3ddaa71a}


Method that smears the argument 4 vector $\ast$NOTE$\ast$ the particle ID must be set before calling this method. 

\begin{Desc}
\item[Parameters:]
\begin{description}
\item[{\em v}]The 4vector smearing target \end{description}
\end{Desc}
\index{DalitzSmear@{Dalitz\-Smear}!SmearVector@{SmearVector}}
\index{SmearVector@{SmearVector}!DalitzSmear@{Dalitz\-Smear}}
\subsubsection{\setlength{\rightskip}{0pt plus 5cm}TLorentz\-Vector Dalitz\-Smear::Smear\-Vector ()}\label{classDalitzSmear_2461cb6cfeb9885b9fe9ac1c7992a814}


Method that smears a 4 vector $\ast$NOTE$\ast$ v\_\-reg and pid global variables must already be set. 



\subsection{Member Data Documentation}
\index{DalitzSmear@{Dalitz\-Smear}!pid@{pid}}
\index{pid@{pid}!DalitzSmear@{Dalitz\-Smear}}
\subsubsection{\setlength{\rightskip}{0pt plus 5cm}int \bf{Dalitz\-Smear::pid}}\label{classDalitzSmear_c7fd2ffdba0c574177bd5d6de605d7e5}


The local particle ID associated with the current copy of v\_\-reg. 

\index{DalitzSmear@{Dalitz\-Smear}!q@{q}}
\index{q@{q}!DalitzSmear@{Dalitz\-Smear}}
\subsubsection{\setlength{\rightskip}{0pt plus 5cm}double \bf{Dalitz\-Smear::q}}\label{classDalitzSmear_963895d9a71287579aaa90087c607081}


Charge, used only electron(-1) and positron(+1). 

\index{DalitzSmear@{Dalitz\-Smear}!RNG@{RNG}}
\index{RNG@{RNG}!DalitzSmear@{Dalitz\-Smear}}
\subsubsection{\setlength{\rightskip}{0pt plus 5cm}TRandom1$\ast$ \bf{Dalitz\-Smear::RNG}}\label{classDalitzSmear_bfb2d568818dbaa7db9936a52bd9c079}


Local copy of random number generator used for generating gaussian values. 

\index{DalitzSmear@{Dalitz\-Smear}!scaleParameterCP@{scaleParameterCP}}
\index{scaleParameterCP@{scaleParameterCP}!DalitzSmear@{Dalitz\-Smear}}
\subsubsection{\setlength{\rightskip}{0pt plus 5cm}double \bf{Dalitz\-Smear::scale\-Parameter\-CP}\hspace{0.3cm}{\tt  [private]}}\label{classDalitzSmear_1ddd88ddea64ea384154ddf4a60d5574}


The scale parameter sigma for a charged particle, for drawing random deviates from rayleigh distribution. 

\index{DalitzSmear@{Dalitz\-Smear}!scaleParameterNP@{scaleParameterNP}}
\index{scaleParameterNP@{scaleParameterNP}!DalitzSmear@{Dalitz\-Smear}}
\subsubsection{\setlength{\rightskip}{0pt plus 5cm}double \bf{Dalitz\-Smear::scale\-Parameter\-NP}\hspace{0.3cm}{\tt  [private]}}\label{classDalitzSmear_7449e92c38450dc5d233046e9866100c}


The scale parameter sigma for a neutral particle, for drawing random deviates from rayleigh distribution. 

\index{DalitzSmear@{Dalitz\-Smear}!v_reg@{v\_\-reg}}
\index{v_reg@{v\_\-reg}!DalitzSmear@{Dalitz\-Smear}}
\subsubsection{\setlength{\rightskip}{0pt plus 5cm}TLorentz\-Vector \bf{Dalitz\-Smear::v\_\-reg}}\label{classDalitzSmear_9340d830c97814a02c2b2749e924abc8}


Locally stored 4 vector which is unsmeared. 

\index{DalitzSmear@{Dalitz\-Smear}!v_sme@{v\_\-sme}}
\index{v_sme@{v\_\-sme}!DalitzSmear@{Dalitz\-Smear}}
\subsubsection{\setlength{\rightskip}{0pt plus 5cm}TLorentz\-Vector \bf{Dalitz\-Smear::v\_\-sme}}\label{classDalitzSmear_87df42c7420bd96bcd61ce25d10ec6ed}


Locally stored 4 vector which is the smeared version of v\_\-reg. 



The documentation for this class was generated from the following file:\begin{CompactItemize}
\item 
/home/justin/work/Single\-Particle\-Generator/dalitz\_\-decay/Smearing/\bf{Dalitz\-Smear.h}\end{CompactItemize}
